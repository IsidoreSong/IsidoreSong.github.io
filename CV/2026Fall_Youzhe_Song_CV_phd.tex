\documentclass[10pt, letterpaper]{article}

% Packages:
\usepackage[
    ignoreheadfoot, % set margins without considering header and footer
    top=1 cm, % seperation between body and page edge from the top
    bottom=0.4 cm, % seperation between body and page edge from the bottom
    left=1 cm, % seperation between body and page edge from the left
    right=1 cm, % seperation between body and page edge from the right
    footskip=1.0 cm, % seperation between body and footer
    % showframe % for debugging
]{geometry} % for adjusting page geometry
\usepackage{titlesec} % for customizing section titles
\usepackage{tabularx} % for making tables with fixed width columns
\usepackage{array} % tabularx requires this
\usepackage[dvipsnames]{xcolor} % for coloring text
\definecolor{primaryColor}{RGB}{0, 79, 144} % define primary color
\usepackage{enumitem} % for customizing lists
\usepackage{fontawesome5} % for using icons
\usepackage{amsmath} % for math
\usepackage[
    pdftitle={Youzhe Song's CV},
    pdfauthor={Youzhe Song},
    pdfcreator={LaTeX with RenderCV},
    colorlinks=true,
    urlcolor=primaryColor
]{hyperref} % for links, metadata and bookmarks
\usepackage[pscoord]{eso-pic} % for floating text on the page
\usepackage{calc} % for calculating lengths
\usepackage{bookmark} % for bookmarks
\usepackage{lastpage} % for getting the total number of pages
\usepackage{changepage} % for one column entries (adjustwidth environment)
\usepackage{paracol} % for two and three column entries
\usepackage{ifthen} % for conditional statements
\usepackage{needspace} % for avoiding page brake right after the section title
\usepackage{iftex} % check if engine is pdflatex, xetex or luatex
% 优先使用 XCharter(粗体更明显),不存在则回退 charter
\IfFileExists{XCharter.sty}{%
    \usepackage{XCharter}
}{%
    \usepackage{charter}
}

% Ensure that generate pdf is machine readable/ATS parsable:
\ifPDFTeX
    \input{glyphtounicode}
    \pdfgentounicode=1
    \usepackage[T1]{fontenc}
    \usepackage[utf8]{inputenc}
    \usepackage{lmodern}
\fi

% Some settings:
\raggedright
\AtBeginEnvironment{adjustwidth}{\partopsep0pt} % remove space before adjustwidth environment
\pagestyle{empty} % no header or footer
\setcounter{secnumdepth}{0} % no section numbering
\setlength{\parindent}{0pt} % no indentation
\setlength{\topskip}{0pt} % no top skip
\setlength{\columnsep}{0.15cm} % set column seperation

\titleformat{\section}{\needspace{4\baselineskip}\bfseries\large}{}{0pt}{}[\vspace{1pt}\titlerule]

\titlespacing{\section}{
    % left space:
    -1pt
}{
    % top space:
    0.25 cm
}{
    % bottom space:
    0.15 cm
} % section title spacing

\renewcommand\labelitemi{$\bullet$} % custom bullet points
\newenvironment{highlights}{
    \begin{itemize}[
        topsep=0.05 cm,
        parsep=0.05 cm,
        partopsep=0pt,
        itemsep=0pt,
        leftmargin=0.4 cm + 10pt
    ]
}{
    \end{itemize}
} % new environment for highlights

\newenvironment{onecolentry}{
    \begin{adjustwidth}{
        0.2 cm + 0.00001 cm
    }{
        0.2 cm + 0.00001 cm
    }
}{
    \end{adjustwidth}
} % new environment for one column entries

\newenvironment{twocolentry}[2][]{
    \onecolentry
    \def\secondColumn{#2}
    \setcolumnwidth{\fill, 4.5 cm}
    \begin{paracol}{2}
}{
    \switchcolumn \raggedleft \secondColumn
    \end{paracol}
    \endonecolentry
} % new environment for two column entries

\newenvironment{header}{
    \setlength{\topsep}{0pt}\par\kern\topsep\centering\linespread{1.2}
}{
    \par\kern\topsep
} % new environment for the header

% save the original href command in a new command:
\let\hrefWithoutArrow\href

\begin{document}
    \newcommand{\AND}{\unskip
        \cleaders\copy\ANDbox\hskip\wd\ANDbox
        \ignorespaces
    }
    \newsavebox\ANDbox
    \sbox\ANDbox{\quad|\quad}

    \begin{header}
        \textbf{\fontsize{24 pt}{24 pt}\selectfont Youzhe Song}
        \\[2pt]
        \normalsize
        \mbox{\href{https://isidoresong.github.io/}{\color{black}{\footnotesize\faHome}\hspace*{0.13cm}Homepage}}%
        \AND%
        \kern 0.25 cm%
        \mbox{{\color{black}\footnotesize\faMapMarker*}\hspace*{0.13cm}Shanghai, China}%
        \kern 0.25 cm%
        \AND%
        \kern 0.25 cm%
        \mbox{\hrefWithoutArrow{mailto:yanfengz@outlook.com}{\color{black}{\footnotesize\faEnvelope[regular]}\hspace*{0.13cm}yanfengz@outlook.com}}%
        \kern 0.25 cm%
        \AND%
        \kern 0.25 cm%
        \mbox{\hrefWithoutArrow{tel:+8618019236930}{\color{black}{\footnotesize\faPhone*}\hspace*{0.13cm}+86 180 1923 6930}}%
        \kern 0.25 cm%
    \end{header}

    \section{Education}
        \begin{twocolentry}{
            \textit{Sept 2021 – June 2024}}
            \textbf{East China Normal University} (Project 985)

            \textit{M.S. in Computer Science and Technology}
        \end{twocolentry}
        \begin{onecolentry}
            \begin{highlights}
                \item GPA: 3.77/4.00; Admitted via recommendation (waiver of entrance exam)
                \item \textbf{Research Focus:} Visual Generalization, Contrastive Learning (Advisor: Feng Wang, Associate Researcher)
            \end{highlights}
        \end{onecolentry}

        \begin{twocolentry}{
            \textit{Sept 2017 – June 2021}}
            \textbf{Donghua University} (Project 211)

            \textit{B.S. in Computer Science and Technology}
        \end{twocolentry}
        \begin{onecolentry}
            \begin{highlights}
                \item GPA: 4.01/4.30; \textbf{Rank: 2/180}
                \item \textbf{Honors:} Outstanding Graduate, University Scholarship
            \end{highlights}
        \end{onecolentry}
        
        \begin{twocolentry}{
            \textit{Aug 2019 – Aug 2020}}
            \textbf{Purdue University Northwest}

            \textit{Exchange Student \& Research Assistant, Computer and Information Technology}
        \end{twocolentry}
        \begin{onecolentry}
            \begin{highlights}
                \item GPA: 3.90/4.00; Research Assistant under Prof. Keyuan Jiang (Department Chair)
                \item Awarded full CSC scholarship; \textbf{1st Place}, Notre Dame Data Hackathon
            \end{highlights}
        \end{onecolentry}
        
    \section{Publications}
        \begin{onecolentry}
            \begin{enumerate}[
                label=\arabic*.,
                topsep=0pt,
                parsep=0pt,
                partopsep=0pt,
                itemsep=0.1cm,
                leftmargin=*
            ]
                \item \textbf{Youzhe Song}, Feng Wang. "CoReFace: Sample-guided Contrastive Regularization for Deep Face Recognition." \textit{Pattern Recognition}, vol. 152, art. 110483, 2024.
                \href{https://doi.org/10.1016/j.patcog.2024.110483}{\faLink}
                \href{https://github.com/IsidoreSong/CoreFace}{\faGithub}
                
                \item \textbf{Youzhe Song}, Feng Wang. "QGFace: Quality-Guided Joint Training For Mixed-Quality Face Recognition." \textit{In Proceedings of the 2024 IEEE International Conference on Automatic Face and Gesture Recognition (FG)}.
                \href{https://arxiv.org/abs/2312.17494}{\faLink}
                \href{https://github.com/IsidoreSong/QGFace}{\faGithub}
            \end{enumerate}
        \end{onecolentry}

    \section{Research Experience}
        \begin{onecolentry}
            \textbf{CoReFace: Improving Generalization with Contrastive Regularization}
            \begin{highlights}
                \item \textbf{Identified model brittleness} in open-set recognition as a key challenge to generalization.
                \item Proposed \textbf{CoReFace}, a framework using \textbf{contrastive learning} as a regularizer to learn a more discriminative feature space, \textbf{increasing the similarity margin for positive/negative pairs by 15\%}.
                \item Achieved state-of-the-art performance by applying self-supervised principles to supervised learning, offering a new path for building more robust recognition models.
            \end{highlights}
        \end{onecolentry}

        \begin{onecolentry}
            \textbf{QGFace: Unified Representation Learning for Heterogeneous Data}
            \begin{highlights}
                \item \textbf{Addressed the critical, real-world issue} of training a single model on heterogeneous-quality data.
                \item Designed a \textbf{unified, single-encoder} with a quality-aware strategy, improving efficiency and scalability.
                \item Reached state-of-the-art on mixed-quality benchmarks with a practical solution for robust recognition.
            \end{highlights}
        \end{onecolentry}

% --- START OF NEW Experience SECTION CODE ---
\section{Industry Experience}
    \begin{twocolentry}{
        \textit{June 2024 – Present}
    }
        \textbf{Technical Manager}
        
        \textit{Worldwide Educators Conference (\href{https://www.wwec820.com/}{WWEC})}
        
    \end{twocolentry}
    \begin{onecolentry}
        \begin{highlights}
            \item Managed internal and external technical associations for this \textbf{10K-participant conference} and co-piloted with ai to develope an automated tool to generate 1K+ custom assets, saving an estimated 100+ hours of manual design work.
            \item Stimulated our team to digitize the storage of data and the pipeline to further support the growth of this project.
            \item Established the official website with repeated discussions among colleagues from multiple departments. \href{https://www.wwec820.com}{\faLink}
        \end{highlights}
    \end{onecolentry}

    \vspace{0.1cm}

    \begin{twocolentry}{
        \textit{May 2023 – Oct 2023}
    }
        \textbf{AI Full-Stack Developer Intern}
        
        \textit{Shanghai KeyoneAI Technology Co., Ltd.}
    \end{twocolentry}
    \begin{onecolentry}
        \begin{highlights}
            \item \textbf{Supervisor:} \href{https://www.linkedin.com/in/jie-fang-28293740}{Jie Fang (former Chief AI Architect, IBM China)}
            \item Delivered a powerful design tool by engineering a full-stack application; fine-tuned \textbf{Stable Diffusion} with \textbf{LoRA} and \textbf{ControlNet} to enable intuitive, controllable features like text-to-image and inpainting.
               \end{highlights}
    \end{onecolentry}

    \vspace{0.1cm}

    \begin{twocolentry}{
        \textit{May 2021 – Aug 2021}
    }
        \textbf{AI Algorithm Intern}
        
        \textit{Shanghai Feifan Technology Co., Ltd.}
    \end{twocolentry}
    \begin{onecolentry}
        \begin{highlights}
            \item \textbf{Supervisor:} \href{https://www.linkedin.com/in/jie-fang-28293740}{Jie Fang (former Chief AI Architect, IBM China)}
            \item Developed a fashion aesthetics recommendation system leading to a \textbf{20\% inventory reduction} for the client.
            % \item Built the core of the system by designing a color block classification model and establishing a hierarchical rule-based engine to generate aesthetic clothing combinations.            % \item Initiated the project's data pipeline from scratch, implementing web scraping, ETL processes.
        \end{highlights}
    \end{onecolentry}
% --- END OF NEW Experience SECTION CODE ---
    
    \section{Skills}
        \begin{onecolentry}
            \textbf{Languages:} Python, JavaScript
        \end{onecolentry}
        \begin{onecolentry}
            \textbf{Frameworks \& Tools:} PyTorch, PyTorch Lightning, Docker, WandB, Gradio
        \end{onecolentry}
        \begin{onecolentry}
            \textbf{Systems:} HPC Cluster, Server Maintenance
        \end{onecolentry}
\end{document}