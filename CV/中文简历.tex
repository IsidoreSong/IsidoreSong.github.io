% !TEX TS-program = XeLaTex
\documentclass[10pt, letterpaper]{article}
% \usepackage{lwarp}

% Packages:
\usepackage[
    ignoreheadfoot, % set margins without considering header and footer
    top=1 cm, % seperation between body and page edge from the top
    bottom=1 cm, % seperation between body and page edge from the bottom
    left=2 cm, % seperation between body and page edge from the left
    right=2 cm, % seperation between body and page edge from the right
    footskip=1.0 cm, % seperation between body and footer
    % showframe % for debugging
]{geometry} % for adjusting page geometry
\usepackage{titlesec} % for customizing section titles
\usepackage{tabularx} % for making tables with fixed width columns
\usepackage{array} % tabularx requires this
\usepackage[dvipsnames]{xcolor} % for coloring text
\definecolor{primaryColor}{RGB}{0, 0, 0} % define primary color
\usepackage{enumitem} % for customizing lists
\usepackage{fontawesome5} % for using icons


% Packages:

\usepackage{amsmath} % for math
\usepackage[
    pdftitle={宋有哲的简历},
    pdfauthor={宋有哲},
    pdfcreator={LaTeX with RenderCV},
    colorlinks=true,
    urlcolor=primaryColor
]{hyperref} % for links, metadata and bookmarks
\usepackage[pscoord]{eso-pic} % for floating text on the page
\usepackage{calc} % for calculating lengths
\usepackage{bookmark} % for bookmarks
\usepackage{lastpage} % for getting the total number of pages
\usepackage{changepage} % for one column entries (adjustwidth environment)
\usepackage{paracol} % for two and three column entries
\usepackage{ifthen} % for conditional statements
\usepackage{needspace} % for avoiding page brake right after the section title
\usepackage{iftex} % check if engine is pdflatex, xetex or luatex
\usepackage{fontspec} % Required for XeLaTeX/LuaLaTeX

% Font setup for Chinese characters
\ifdefined\HCode\else\usepackage{xeCJK}\fi
% \setmainfont{Noto Serif CJK SC}

% Ensure that generate pdf is machine readable/ATS parsable:
\ifPDFTeX
    \input{glyphtounicode}
    \pdfgentounicode=1
    \usepackage[T1]{fontenc}
    \usepackage[utf8]{inputenc}
    \usepackage{lmodern}
\fi


% Some settings:
\raggedright
\AtBeginEnvironment{adjustwidth}{\partopsep0pt} % remove space before adjustwidth environment
\pagestyle{empty} % no header or footer
\setcounter{secnumdepth}{0} % no section numbering
\setlength{\parindent}{0pt} % no indentation
\setlength{\topskip}{0pt} % no top skip
\setlength{\columnsep}{0.15cm} % set column seperation
\pagenumbering{gobble} % no page numbering

\titleformat{\section}{\needspace{4\baselineskip}\bfseries\large}{}{0pt}{}[\vspace{1pt}\titlerule]

\titlespacing{\section}{
    % left space:
    -1pt
}{
    % top space:
    0.3 cm
}{
    % bottom space:
    0.2 cm
} % section title spacing

\renewcommand\labelitemi{$\bullet$} % custom bullet points
\newenvironment{highlights}{
    \begin{itemize}[
        topsep=0.05 cm,
        parsep=0.05 cm,
        partopsep=0pt,
        itemsep=0pt,
        leftmargin=0.4 cm + 10pt
    ]
}{
    \end{itemize}
} % new environment for highlights


\newenvironment{highlightsforbulletentries}{
    \begin{itemize}[
        topsep=0.05 cm,
        parsep=0.05 cm,
        partopsep=0pt,
        itemsep=0pt,
        leftmargin=0.4 cm + 10pt
    ]
}{
    \end{itemize}
} % new environment for highlights for bullet entries

\newenvironment{onecolentry}{
    \begin{adjustwidth}{
        0 cm + 0.00001 cm
    }{
        0 cm + 0.00001 cm
    }
}{
    \end{adjustwidth}
} % new environment for one column entries

\newenvironment{twocolentry}[2][]{
    \onecolentry
    \def\secondColumn{#2}
    \setcolumnwidth{\fill, 4.5 cm}
    \begin{paracol}{2}
}{
    \switchcolumn \raggedleft \secondColumn
    \end{paracol}
    \endonecolentry
} % new environment for two column entries

\newenvironment{threecolentry}[3][]{
    \onecolentry
    \def\thirdColumn{#3}
    \setcolumnwidth{, \fill, 4.5 cm}
    \begin{paracol}{3}
    {\raggedright #2} \switchcolumn
}{
    \switchcolumn \raggedleft \thirdColumn
    \end{paracol}
    \endonecolentry
} % new environment for three column entries

\newenvironment{header}{
    \setlength{\topsep}{0pt}\par\kern\topsep\centering\linespread{1.5}
}{
    \par\kern\topsep
} % new environment for the header

\newcommand{\placelastupdatedtext}{% \placetextbox{<horizontal pos>}{<vertical pos>}{<stuff>}
  \AddToShipoutPictureFG*{% Add <stuff> to current page foreground
    \put(
        \LenToUnit{\paperwidth-2 cm-0 cm+0.05cm},
        \LenToUnit{\paperheight-1.0 cm}
    ){\vtop{{\null}\makebox[0pt][c]{
        \small\color{gray}\textit{Last updated in September 2024}\hspace{\widthof{Last updated in September 2024}}
    }}}%
  }%
}%

% save the original href command in a new command:
\let\hrefWithoutArrow\href


\begin{document}
    \newcommand{\AND}{\unskip
        \cleaders\copy\ANDbox\hskip\wd\ANDbox
        \ignorespaces
    }
    \newsavebox\ANDbox
    \sbox\ANDbox{\quad|\quad}

    \begin{header}
        \fontsize{25 pt}{25 pt}\selectfont 宋有哲

        \vspace{5 pt}

    \normalsize
    \mbox{\href{https://isidoresong.github.io/}{\faHome\space Homepage}}%
    % \mbox{\href{https://isidoresong.github.io/}{\color{black}{\footnotesize\faHome}\hspace*{0.13cm}Homepage}}%
    \AND \mbox{\hrefWithoutArrow{mailto:yanfengz@outlook.com}{\faEnvelope\space yanfengz@outlook.com}
    \AND \hrefWithoutArrow{tel:+8618019236930}{\faPhone\space +86 180 1923 6930}}% <-- 请将 +86 123-4567-8910 替换为您的电话号码
    % \\ % 换行
    % --- 联系信息第二行:地址 ---
    \AND\mbox{\faMapMarker\space 上海市浦东软件园}%
    \end{header}

    \vspace{5 pt - 0.3 cm}

    % \section{个人简介}
    % \begin{onecolentry}
    % 	\parbox{\linewidth}{
	% 	我对构建通用且鲁棒的基础模型充满热情。我专注于通过自监督对比学习提升模型的底层学习和泛化能力。我具备从零开始快速掌握并部署AI系统的全栈能力。我渴望在博士阶段继续探索如何让模型理解真实世界与概念之间关系,为构建更可信赖的基础模型做出贡献。
	% 	}
    % \end{onecolentry}


    \section{教育背景}

        \begin{twocolentry}{
            2021年09月 – 2024年06月
        }
            \textbf{华东师范大学 (985)} | 硕士, 计算机科学与技术
        \end{twocolentry}
        \vspace{0.10 cm}
        \begin{onecolentry}
            \begin{highlights}
                \item GPA: 3.77/4.00, \textbf{直研保送}
                \item \textbf{研究方向:}自监督表示学习、视觉模型的泛化能力、面向开放世界数据的对比学习。
            \end{highlights}
        \end{onecolentry}

        \vspace{0.2 cm}
        
        \begin{twocolentry}{
            2017年09月 – 2021年06月
        }
            \textbf{东华大学 (211)} | 本科, 计算机科学与技术
        \end{twocolentry}
        \vspace{0.10 cm}
        \begin{onecolentry}
            \begin{highlights}
                \item GPA: 4.00/4.30, \textbf{专业排名: 2/180}
                \item \textbf{荣誉:} \textbf{优秀毕业生 (Outstanding Graduate)}, 东华大学奖学金, 恒逸一等奖学金
            \end{highlights}
        \end{onecolentry}
        
        \vspace{0.2 cm}

        \begin{twocolentry}{
            2019年08月 – 2020年08月
        }
            \textbf{普渡大学 (Purdue University)} | 交换生与科研助理, 计算机通信技术
        \end{twocolentry}
        \vspace{0.10 cm}
        \begin{onecolentry}
            \begin{highlights}
                \item GPA: 3.9/4.0
                \item 获国家留学基金委全额奖学金;\textbf{冠军 (1st Place)},美国圣母大学 Data Hackathon。
            \end{highlights}
        \end{onecolentry}
        
    \section{学术出版 (Publications)}

\begin{onecolentry}
    \begin{enumerate}[
        label=\arabic*.,
        topsep=0pt,
        parsep=0pt,
        partopsep=0pt,
        itemsep=0.2cm,
        leftmargin=*
    ]
        \item \textbf{Youzhe Song}, Feng Wang. CoReFace: Sample-guided Contrastive Regularization for Deep Face Recognition.
        Pattern Recognition 152 (2024): 110483.
        \href{https://doi.org/10.1016/j.patcog.2024.110483}{\faLink}
        \href{https://github.com/IsidoreSong/CoreFace}{\faGithub}
        
        \item \textbf{Youzhe Song}, Feng Wang. QGFace: Quality-Guided Joint Training For Mixed-Quality Face Recognition.
        In Proceedings of
the 2024 IEEE International Conference on Automatic Face and Gesture Recognition. \href{https://arxiv.org/abs/2312.17494}{\faLink}
        \href{https://github.com/IsidoreSong/QGFace}{\faGithub}
    \end{enumerate}
\end{onecolentry}

    \section{科研经历 (Research Experience)}
        \begin{onecolentry}
            \textbf{CoReFace: 通过对比正则化改善泛化能力}
            \begin{highlights}
                \item \textbf{识别出开放集识别中的模型脆弱性}是泛化的关键挑战。
                \item 提出\textbf{CoReFace}框架,使用\textbf{对比学习}作为正则化器,学习更具辨别力的特征空间,\textbf{将正/负样本对的相似度边距提高了15\%}。
                \item 通过将自监督原则应用于监督学习,实现了最先进的性能,为构建更鲁棒的识别模型提供了新途径。
            \end{highlights}
        \end{onecolentry}
        \vspace{0.2 cm}
        \begin{onecolentry}
            \textbf{QGFace: 异构数据的统一表示学习}
            \begin{highlights}
                \item \textbf{解决了在异构质量数据上训练单个模型的关键现实问题}。
                \item 设计了带有质量感知策略的\textbf{统一单编码器},提高了效率和可扩展性。
                \item 在混合质量基准上达到了最先进水平,为鲁棒识别提供了实用的解决方案。
            \end{highlights}
        \end{onecolentry}

\section{工作与实习经历}

    \begin{onecolentry}
        \large\textbf{世界教育者大会 (Worldwide Educators Conference)},上海
    \end{onecolentry}
    \begin{twocolentry}{
        2024年06月 – 今
    }
        \textbf{技术经理}
    \end{twocolentry}
    \begin{onecolentry}
        \begin{highlights}
            \item 协调\textbf{万人会议}会议管理内外部技术需求与对接,借助AI高效开发自动化工具,生成\textbf{超过1000个}定制资产,近似节省了\textbf{超100小时}的人工设计工作。管理域名、网站和服务器的日常维护。
            \item 推动团队数据存储与工作流程数字化,以进一步支持该项目的持续增长。
            \item 与多部门同事反复讨论,确立内容,建立官方网站。\href{https://www.wwec820.com}{\faLink}
        \end{highlights}
    \end{onecolentry}
    
    \vspace{0.2 cm}

    \begin{onecolentry}
         \large\textbf{上海启元科技有限公司}, 上海
    \end{onecolentry}
    \begin{twocolentry}{
        2023年05月 – 2023年10月
    }
       \textbf{全栈开发实习生}
    \end{twocolentry}
    \begin{onecolentry}
        \begin{highlights}
            \item \textbf{导师 (Supervisor):} \href{https://www.linkedin.com/in/jie-fang-28293740}{方杰 (前IBM中国首席AI架构师, 前工商银行AI实验室负责人)}
            \item 通过设计一个全栈应用程序,交付了一款功能强大的设计工具;利用LoRA和ControlNet对Stable Diffusion进行了微调,以实现直观、可控的功能,如文本生成图像和图像修复。
        \end{highlights}
    \end{onecolentry}

    \vspace{0.2 cm}
    
    \begin{onecolentry}
        \large\textbf{上海斐樊科技有限公司}, 上海
    \end{onecolentry}
    \begin{twocolentry}{
        2021年05月 – 2021年08月
    }
        \textbf{智能算法实习生}
    \end{twocolentry}
    \begin{onecolentry}
        \begin{highlights}
            \item \textbf{导师 (Supervisor):} \href{https://www.linkedin.com/in/jie-fang-28293740}{方杰 (前IBM中国首席AI架构师, 前工商银行AI实验室负责人)}
            \item 开发了一个时尚美学搭配推荐系统,为客户减少了 20\% 的库存。
        \end{highlights}
    \end{onecolentry}

    \section{专业技能}
        \begin{onecolentry}
            \textbf{编程语言:} Python, JavaScript
        \end{onecolentry}
        \begin{onecolentry}
            \textbf{框架与工具:} PyTorch, PyTorch Lightning, Docker, WandB, Gradio
        \end{onecolentry}
        \begin{onecolentry}
            \textbf{系统:} HPC集群, 服务器维护
        \end{onecolentry}

\end{document}